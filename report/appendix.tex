\appendix
\chapter{Appendix}
\section{Proposed design}
In this section we show the detailed ER diagram of our proposed design. The ER diagram has been split into four different figures for better readability. We have also discussed the new tables that were created to improve customer relationship management.
\begin{figure}
\centering
\includegraphics[height=11cm,width=15cm]{images/trans1}
\caption{Purchase Tables}
\label{Pdesign1}
\end{figure}
\begin{figure}
\centering
\includegraphics[height=11cm,width=15cm]{images/profile_tables1}
\caption{Profile Tables}
\label{Pdesign2}
\end{figure}
\begin{figure}
\centering
\includegraphics[height=10cm,width=16cm]{images/summary1}
\caption{Summary Tables}
\label{Pdesign3}
\end{figure}
\begin{figure}
\centering
\includegraphics[height=10cm,width=14cm]{images/survey1}
\caption{Survey Tables}
\label{Pdesign4}
\end{figure}
\subsection{New Tables Schemas}
\textbf{Vendor: }The client related information are stored in this table.\\
\textbf{Vendor\_visit: }In this table, the \emph{user\_id} attribute records the customer information and the \emph{vendor\_id} attribute records the corresponding client information he/she visited.\\
\textbf{User\_product\_interaction: }This table captures the customer information and the corresponding product information he/she had shown interest on. A record is inserted into the table every time a customer interacts with a product. The attribute \emph{start\_time} captures the time when an interaction begins and the attribute \emph{end\_time} captures the time when the interaction ends. We can use the time difference (end\_time - start\_time) to extract the level of interest of a customer on a particular product (the higher the time difference, then higher the customer's interest on the product).\\
\textbf{Customer\_purchases: }This table contains the customer information (such as customer's id) and the corresponding product information he/she purchased (product id, amount spent on the product). This table also contains the date and time of each purchase. A record is inserted into the table every time a customer purchases a product.\\
\textbf{Product\_groups: }Products are classified into groups and a unique \emph{group\_id} is assigned to each group. \emph{Product\_groups} table combined with \emph{Customer\_purchases} table enables the organization to market other products in the group based on the customer purchase pattern. For example, the electronics group contain products such as phones, laptops and tablets. If a customer had purchased a phone (information extracted from the transactions table) the organization can  market laptops and tablets to the same customer (as laptops and tablets belongs to the same electronics group).\\\\
The Summary tables contain aggregated information extracted from the above mentioned tables. This enhances the analysis process (e.g., total sales of a product, most profitable customer), as  the organization need not combine multiple tables or run analytical functions (such as sum, difference and total count) to extract information. They can execute a selection query on the summary tables to extract the above mentioned highlights. \texttt{Figure \ref{Pdesign3}.User\_summary} for each customer contains, the total products purchased, the number of clients he/she visited and the total time he/she spent on a product. \texttt{Figure \ref{Pdesign3}.Vendor\_event\_summary} for each client contains, total sales, total time spent by the customers and the number of customers that visited. \texttt{Figure \ref{Pdesign3}.Product\_summary} for each product contains, total quantity sold, the number of clients interacted and total time spent by the customers. The products are classified into groups and \texttt{Figure \ref{Pdesign3}.Group\_summary} for each group contains, total time and amount spent by the customers.
\newpage
\section{Implemented Queries}
{\Large\textbf{SALES QUERIES}\par}
\begin{lstlisting}
--1) Total sales across all events
SELECT name AS event_name,SUM(amount) AS Total_event_sales
FROM transactions,event
GROUP BY name;


--2) Total sales across all clients for one event
SELECT name AS event_name,name AS clients_name,transaction_date,SUM(amount) AS total_amount
FROM transactions,event,client
WHERE event id = '957'
GROUP BY name,transaction date;


--3) Client and Customer details for products purchased at an event
SELECT id,product_name,name AS clients_name,name AS event_name,first_name
FROM product_transactions,products,clients,customer
WHERE product_id IN
(SELECT product_id
FROM  product_transactions, transactions
WHERE event_id = '957'
GROUP BY product_id, clients_id)
AND id = '957';


--4) Identify the top-5 sold products (in terms of quantity sold) at a given event.  Return the product id and name for each product.
SELECT id, product_name
FROM product
WHERE id IN
(SELECT product_id
FROM  product_transactions,transactions
WHERE event_id = '957'
GROUP BY product_id
ORDER BY count(product_id) DESC
FETCH FIRST 5 ROWS ONLY);
\end{lstlisting}
{\Large \textbf{CUSTOMER QUERIES}}
\begin{lstlisting}
--5) For each customer-to-product interaction at a given client, return the customer's product rating and the customer feedback.
SELECT DISTINCT question, answer, customer_id
FROM survey answers,questions,answers
WHERE question_id BETWEEN 49 AND 109 AND customer_id IN
(SELECT customer_id
FROM customer_product_interaction
WHERE  Clients_id = 1 );


--6) Return the top-5 products according to men (measured by their interaction time).
SELECT id, product_name
FROM product
WHERE id IN
(SELECT product_id
FROM customer_product_interaction,survey_answers
WHERE question_id = '33' AND answer_id = '39'GROUP BY
product_id ORDER BY
COUNT(product_id) DESC
FETCH FIRST 5 ROWS ONLY;


--7) ROI per client across all events
SELECT DISTINCT id, first_name, Clients_id, name AS Clients_name, SUM(amount) AS transaction sum
FROM ps_behavior_profile_500K,customer,clients
WHERE transaction_id IS NOT NULL AND customer_id IN
(SELECT customer_id
FROM transactions t)
GROUP BY id, first_name, Clients_id, name
ORDER BY transaction sum DESC;


--8) Identify all products interacted and later purchased by a given customer
SELECT id AS ProdID Interacted, prod_name AS ProdName Interacted,id AS ProdID purchased, prod_name AS ProdName Purchased
FROM (SELECT DISTINCT product_id AS id, product_name AS prod_name FROM customer_product_interaction,product
WHERE customer_id = 232 AND product_id NOT IN
(SELECT product_id FROM transactions,product_transactions
WHERE customer_id = 232)) temp1
,(SELECT DISTINCT product_id AS id, p1.product_name AS prod_name  FROM customer_product_interaction, product
WHERE customer_id = 232 AND product_id IN
(SELECT product_id FROM transactions,product_transactions
WHERE customer_id = 232)) temp2;
\end{lstlisting}
{\Large\textbf{AGGREGATION QUERIES}}
\begin{lstlisting}
--9) For each potential customer, return client name and product name he/she expressed interest
SELECT id,first_name,last_name,name AS Clients_name,product_name,Time_spent_product
FROM customer_product_interaction,product,clients
WHERE product_id = id AND id IN
(SELECT customer_id
FROM customer_product_interaction
WHERE NOT EXISTS
(SELECT 'X'
FROM transactions
WHERE customer_id = customer_id ));
\end{lstlisting}
{\Large\textbf{OVERLAPPING QUERY GROUPS}}\\
\noindent\rule{12.2cm}{0.4pt}\\
{\large\textbf{SALES AGGREGATION QUERIES}}\\\\
The following two queries return most popular products in terms of customer interaction time. Query 10A) runs over warehouse data and Query 10B) runs over transactional data. We compare the running time of both the queries.
\begin{lstlisting}

--10A)
SELECT product_id, product_name AS product_name, SUM(Time_spent_product)
FROM behavior_profile,product
GROUP BY product_id, product_name
ORDER BY SUM(Time_spent_product) DESC;


--10B)
SELECT product_id, product_name AS product_name, SUM(Time spent_Product)
FROM customer_product_interaction,product
GROUP BY product_id,product_name
ORDER BY SUM(Time_spent_Product) DESC;
\end{lstlisting}
{\large\textbf{CUSTOMER AGGREGATION QUERIES}}\\\\
The following two queries return client name and product name each potential customer expressed interest on. Query 11A) runs over warehouse data and Query 11B) runs over transactional data. We compare the running time of both the queries.
\begin{lstlisting}
--11A)
SELECT id, first_name, last_name, email, name AS Clients_name, product_name, Time_spent_product
FROM product,Clients
AND id IN
(SELECT customer_id
FROM behavior_profile
WHERE NOT EXISTS
(SELECT 'X'
FROM transactions
WHERE customer_id = customer_id));


--11B)
SELECT id, first_name, last_name, email, name AS Clients_name, product_name, Time_spent_Product
FROM product,Clients
AND id IN
(SELECT customer_id
FROM customer_product_interaction
WHERE NOT EXISTS
(SELECT 'X'
FROM transactions
WHERE customer_id = customer_id));
\end{lstlisting}
{\large\textbf{COMPARATIVE QUERIES }}
\begin{lstlisting}
--12) Top 5 products across all clients
Original design
select distinct t2.PRODUCT_NAME as pname ,t1.product_id as pid, count(t1.product_id) as Quantity from rogers_check_in t1
inner join rogers_product_list t2 on t1.product_id = t2.id
group by t1.product_id,t2.PRODUCT_NAME
union all
select distinct t3.PRODUCT_NAME as spname ,t4.product_id as spid, count(t4.product_id) as SQuantity from samsung_tap_history t4
inner join samsung_product_list t3 on t4.product_id = t3.id
group by t4.product_id,t3.PRODUCT_NAME
order by Quantity desc
fetch first 5 rows only;

Proposed design
select distinct t2.PRODUCT_NAME,t1.product_id, count(product_id) as quantity from n_user_product_interaction t1
inner join n_product_list t2 on t1.product_id = t2.id
group by t1.product_id,t2.PRODUCT_NAME
order by quantity desc
fetch first 5 rows only;


--13) Uniquely identify customers across all clients and the products they have expressed their interest
Original design
select distinct t1.user_id,t1.first_name,t1.last_name from rogers_user_profile t1
where t1.user_id not in
(select user_id from rogers_check_in t2)
union all
select distinct t1.user_id,t1.first_name,t1.last_name from samsung_user_profile t1
where t1.user_id not in
(select user_id from samsung_tap_history t2);

Proposed design
select distinct t1.id,t1.first_name,t1.last_name from n_user t1
where t1.id not in
(select user_id from n_user_product_interaction t2);


--14) Uniquely identify customers across all clients who want to be notified by clients (Whether customers want clients to contact them)
Original design
select * from rogers_user_profile where contact_consent = 1;

Proposed design
select * from n_user
where id in
(select user_id from n_vendor_visit where contact_consent = 1);


--15) Uniquely identify the survey responses for a given customer across all clients
Original design
select t2.question,t3.answer from rogers_survey_data t1
inner join rogers_questions t2 on t1.question_id = t2.id
inner join rogers_answers t3 on t1.answer_id = t3.id
where user_id in
(select user_id from rogers_user_profile where first_name = 'Ethan' and last_name = 'Do' fetch first row only)
union all
select t2.question,t3.answer from samsung_survey_data t1
inner join samsung_questions t2 on t1.question_id = t2.id
inner join samsung_answers t3 on t1.answer_id = t3.id
where user_id in
(select user_id from samsung_user_profile where first_name = 'Ethan' and last_name = 'Do'fetch first row only);

Proposed design
select t2.question,t3.answer from n_survey_answers t1
inner join n_questions t2 on t1.question_id = t2.id
inner join n_answers t3 on t1.answer_id = t3.id
where user_id in
(select id from n_user where  first_name = 'Ethan' and last_name = 'Do');


--16) Uniquely identify customers who applied for a smart card across all clients
Original design
select id from
(select id from samsung_nfc_list
union all
select id from rogers_nfc_list);

Proposed design
select id from n_nfc_list;

\end{lstlisting} 